\documentclass[%
    fontsize=11pt, % Schriftgröße
    twoside=off, % kein einseitiges Layout
    bibliography=totocnumbered % bibliography has chapter number and shows in the toc
]{scrbook} % Dokumentenklasse: KOMA-Script Book
\usepackage{scrlayer-scrpage} % Anpassbare Kopf- und Fußzeilen

\usepackage[utf8]{inputenc} % Textkodierung: UTF-8
\usepackage[T1]{fontenc} % Zeichensatzkodierung

\usepackage{wrapfig} % Wrapping Images and Text
\usepackage{graphicx} % Grafiken
\usepackage[dvipsnames]{xcolor} % Colors

% Schriftart Helvetica:
\usepackage[scaled]{helvet}
\renewcommand{\familydefault}{\sfdefault}

% Silbentrennung:
\usepackage{hyphenat}
\hyphenation{TUM in-te-res-siert} % Eigene Silbentrennung
%\tolerance 2414
%\hbadness 2414
%\emergencystretch 1.5em
%\hfuzz 0.3pt
%\widowpenalty=10000     % Hurenkinder
%\clubpenalty=10000      % Schusterjungen
%\vfuzz \hfuzz

\usepackage[onehalfspacing]{setspace} % 1,5facher Zeilenabstand
\usepackage{calc} % Berechnungen
\usepackage{enumitem} % Mehr Kontrolle über itemize-, enumerate- und description-Umgebungen
\usepackage{relsize} % Schriftgröße in Abhängigkeit von aktueller anpassen
\usepackage{tabularx} % Flexiblere Tabellen
\usepackage{colortbl} % Tabellen Farbig
\usepackage{caption} % Anpassen von Beschriftungen

% Nummerierung von Abbildungen & Tabellen durchgängig, statt nach Kapiteln:
\usepackage{chngcntr}
\counterwithout{figure}{chapter}
\counterwithout{table}{chapter}

% Abkürzungen, Glossare:
\usepackage[%
    xindy,% xindy zum Indexieren verwenden
    acronym,% Separates Akronym-Verzeichnis
    nopostdot,% Kein Punkt am Ende einer Beschreibung im Glossar
]{glossaries}

% Code render package:
\usepackage{listings}

% Spezielle Befehlsdefinitionen:
\newcommand{\Topic}{}
\newcommand{\savefootnote}[2]{\footnote{\protect\label{#1}#2}} %Save a repeating footnote
\newcommand{\repeatfootnote}[1]{\textsuperscript{\ref{#1}}\normalfont} %Display a repeated footnote

% Links und Fußnoten: 
\usepackage{footnotebackref}
\usepackage{hyperref} % Hyperlinks (needs to be the last package to be imported)

% Debugging:
%\usepackage{showframe} % Layout-Boxen anzeigen
%\usepackage{layout} % Layout-Informationen
%\usepackage{printlen} % Längenwerte ausgeben
